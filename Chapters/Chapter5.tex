\chapter{Les apports du stage}
\label{chap:Chapter 5 title} 
\section*{Introduction}

\hspace{16pt}Le stage chez FCPO a été une expérience riche en apprentissages et en développement professionnel. Ce chapitre examine les différents apports du stage, tant sur le plan théorique qu'intellectuel et pratique. Nous aborderons les compétences techniques acquises, les connaissances théoriques consolidées, et les leçons tirées de l'intégration dans une équipe professionnelle. L'expérience de la gestion de projet et l'application concrète des technologies avancées ont joué un rôle crucial dans notre formation. Ce chapitre vise à offrir une réflexion sur la valeur ajoutée de cette expérience de stage et son impact sur notre développement professionnel.

\pagebreak

\section{Les apports théoriques et intellectuels}

\hspace{16pt} Durant ce stage, j’ai pu approfondir plusieurs concepts théoriques que j'avais appris lors de ma formation académique. Les principaux apports théoriques et intellectuels incluent:

\begin{itemize}
  \item \textbf{Architecture des applications web: }J’ai consolidé mes connaissances sur les architectures MVC (Model-View-Controller) en travaillant avec Symfony et API Platform. J’ai appris à structurer efficacement une application web pour assurer une bonne séparation des préoccupations et une maintenance facile.
  \item \textbf{Systèmes de gestion de bases de données: }J’ai approfondi mes compétences en conception et gestion de bases de données relationnelles, en particulier avec MySQL. J’ai appris à modéliser des entités complexes et à gérer les relations entre elles de manière optimale.
\end{itemize}

\section{Les compétences pratiques développées}

\hspace{16pt}Le stage m’a offert l’opportunité de développer et de perfectionner des compétences pratiques essen-tielles pour un développeur. Les compétences pratiques acquises comprennent:

\begin{itemize}
  \item \textbf{Développement Frontend avec React: } J’ai amélioré mes compétences en développement frontend en utilisant React pour créer des interfaces utilisateur interactives et réactives.
  \item \textbf{Développement Backend avec Symfony et API Platform: }J’ai acquis une expérience pratique significative en utilisant Symfony pour le développement backend et API Platform pour la création d’APIs RESTful. J’ai appris à configurer des opérations CRUD, à valider des champs et à gérer des relations entre entités.
  \item \textbf{Gestion de version avec Git et GitHub: }J’ai perfectionné mes compétences en gestion de version avec Git, en utilisant GitHub pour collaborer avec les membres de l’équipe, gérer des branches, résoudre des conflits et suivre l’historique des modifications.
  \item \textbf{Utilisation d’outils de développement: }J’ai optimisé mon environnement de développement en utilisant Arch Linux, Neovim avec une configuration personnalisée, et des outils comme Alacritty, ZSH, tmux, et fzf. Cela m’a permis de travailler de manière plus efficace et productive.
\end{itemize}

\section{Les apports en termes de monde de l'entreprise}

\hspace{16pt}Ce stage m’a également offert une précieuse perspective sur le fonctionnement et la culture d’entreprise. Voici quelques apports significatifs:

\begin{itemize}
  \item \textbf{Travail en équipe: }J’ai appris à collaborer efficacement avec une équipe de développeurs, à communiquer clairement, à partager des tâches et à résoudre des problèmes ensemble. J’ai compris l’importance de la coordination et de la coopération dans la réussite d’un projet.
  \item \textbf{Gestion de projet: }J’ai été exposé aux méthodologies de gestion de projet agile, telles que les sprints et les réunions quotidiennes. J’ai appris à gérer mon temps, à prioriser les tâches et à respecter les délais tout en assurant la qualité du travail.
  \item \textbf{Adaptation aux exigences des clients: }J’ai acquis une meilleure compréhension de l’importance de répondre aux besoins des clients et de s’adapter rapidement aux changements de spécifications. J’ai appris à être flexible et réactif pour fournir des solutions qui satisfont les attentes des utilisateurs finaux.
  \item \textbf{Culture d’entreprise et éthique professionnelle: }J’ai intégré les valeurs et les pratiques professionnelles de FCPO, telles que le respect des délais, la qualité du travail, la confidentialité des données et l’engagement envers les objectifs de l’entreprise. Cela m’a permis de développer un sens aigu de la responsabilité et de l’éthique professionnelle.
\end{itemize}


\newpage
\section*{Conclusion}

\hspace{16pt}Les apports du stage chez FCPO sont nombreux et variés. Sur le plan théorique, ce stage a permis de consolider les connaissances acquises au cours de nos études, tout en les appliquant à des situations réelles et complexes. Sur le plan pratique, nous avons développé des compétences techniques avancées, appris à naviguer dans un environnement professionnel exigeant, et à collaborer efficacement au sein d'une équipe. Cette expérience a été formatrice à bien des égards, renforçant notre confiance en nos capacités et nous préparant à affronter les défis futurs avec assurance et compétence.
