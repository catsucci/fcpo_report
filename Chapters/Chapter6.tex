\chapter{Les difficultés du stage et les solutions apportées}
\label{chap:Chapter 6 title} 
\section*{Introduction}

\hspace{16pt}Chaque projet ambitieux comporte son lot de défis, et notre stage chez FCPO n'a pas fait exception. Ce chapitre se penche sur les difficultés rencontrées tout au long du développement du chatbot, qu'elles soient techniques, organisationnelles ou relationnelles. Nous analyserons les problèmes spécifiques auxquels nous avons été confrontés et les solutions mises en œuvre pour les surmonter. Cette réflexion vise à démontrer notre capacité à résoudre des problèmes complexes et à adapter nos méthodes de travail en fonction des obstacles rencontrés. En partageant ces expériences, nous mettons en lumière l'importance de la résilience et de l'innovation dans la gestion de projet.

\pagebreak

\section{Les difficultés rencontrées}

\hspace{16pt}Voici les principales difficultés rencontrées au cours de mon stage:

\begin{itemize}
  \item \textbf{Utilisation de nouvelles technologies: }Découvrir et maîtriser des technologies comme Symfony, API Platform et React a été un défi initial majeur, car je n'avais qu'une expérience limitée avec ces outils.
  \item \textbf{Gestion dynamique du chatbot: }Configurer le chatbot pour qu'il réponde de manière dyna-mique et personnalisée aux interactions des utilisateurs a été un défi technique important.
  \item \textbf{Erreurs dues à des drivers manquants: }Des erreurs imprévues causées par des drivers manquants ont perturbé le développement, en particulier lors de l'intégration des fonctionnalités backend.
  \item \textbf{Gestion des domaines: }La configuration et la gestion des domaines pour l'application web ont posé des problèmes, surtout pour assurer une transition fluide entre les environnements de développement et de production.
  \item \textbf{Installation et configuration des versions précises des technologies: }Éviter les con-flits entre différentes versions de bibliothèques et frameworks a nécessité une gestion précise des versions, entraînant des ajustements et des configurations détaillées.
\end{itemize}

\section{Les solutions apportées à ces difficultés}

\hspace{16pt}Des solutions spécifiques ont été mises en place pour résoudre les problèmes.

\begin{itemize}
  \item \textbf{Utilisation de nouvelles technologies: }
    \begin{itemize}
      \item \textbf{Formations et documentation: }J'ai suivi des tutoriels et des formations en ligne pour me familiariser rapidement avec Symfony, API Platform et React. La consultation de la documentation officielle et de forums de développeurs m'a également aidé.
    \end{itemize}
  \item \textbf{Gestion dynamique du chatbot: }
    \begin{itemize}
      \item \textbf{Algorithmes personnalisés: }J'ai implémenté des algorithmes personnalisés, pour main-tenir un flux de travail tout en obtenant le résultat escompté.
    \end{itemize}
  \item \textbf{Erreurs dues à des drivers manquants: }
    \begin{itemize}
      \item \textbf{Installation de packages manquants: }J'ai identifié et installé les drivers manquants grâce à l'analyse des logs d'erreurs. Une vérification systématique des dépendances avant chaque déploiement a été mise en place.
      \item \textbf{Documentation et vérification: }J'ai bien lu une documentation détaillée des prérequis et des configurations nécessaires pour garantir que tous les composants requis étaient correc-tement installés et configurés.
    \end{itemize}
  \item \textbf{Installation et configuration des versions précises des technologies: }
    \begin{itemize}
      \item \textbf{Gestion des dépendances avec Composer et npm: }'ai utilisé Composer pour PHP et npm pour JavaScript pour gérer les dépendances et m'assurer que les versions spécifiques requises étaient installées.
      \item \textbf{Fichiers de configuration détaillés: }Des fichiers de configuration (comme composer.json et package.json) ont été utilisés pour spécifier les versions exactes des bibliothèques et frame-works, réduisant ainsi les risques de conflits et d'incompatibilités.
    \end{itemize}
\end{itemize}


\newpage
\section*{Conclusion}

\hspace{16pt}En conclusion, les difficultés rencontrées durant le stage ont été des opportunités de croissance et d'apprentissage. Les solutions apportées pour surmonter ces défis ont démontré notre capacité à innover et à persévérer face à l'adversité. Ces expériences ont enrichi notre développement professionnel, renforçant notre aptitude à gérer des projets complexes et à collaborer efficacement dans un environnement professionnel dynamique. Les leçons tirées de ces défis nous préparent à affronter les futurs obstacles avec une approche proactive et résiliente.
