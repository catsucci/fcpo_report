\chapter{Les missions effectuées}
\label{chap:Chapter 4 title}
\section*{Introduction}

Lorem ipsum dolor sit amet, consectetur adipiscing elit. Praesent nec dapibus justo. Donec sagittis vulputate ante sed porttitor. Suspendisse sit amet nisl massa. Curabitur nec nisl condimentum, egestas ex vitae, dapibus enim. Etiam iaculis, erat faucibus pellentesque sagittis, nisi justo sollicitudin nibh, et condimentum augue massa non turpis. Proin commodo enim fermentum suscipit condimentum. Maecenas molestie, dui nec vestibulum rhoncus, arcu nisl faucibus neque, a ornare nisi massa ac eros. Aenean id velit sit amet lacus mattis varius. Donec fringilla massa sed nisi eleifend, a aliquet mi tempus. Nunc posuere euismod est, nec tristique augue lobortis non. Sed sodales sem ut metus tempus ullamcorper.
      




\pagebreak

\section{Le Chatbot}

\subsection{Configuration static}

\hspace{16pt}La configuration statique du chatbot concerne les éléments du chatbot qui sont définis à l'avance et qui ne changent pas dynamiquement en fonction des interactions utilisateur. Voici quelques aspects clés de cette configuration:

\begin{itemize}
  \item \textbf{Réponses prédéfinies: }Le chatbot est programmé avec une série de réponses prédéfinies pour les questions fréquentes.
  \item \textbf{Flow du prise de rendez-vous: }Définir un flux statique qui accompagnera l'utilisateur dans sa prise de rendez-vous.
  \item \textbf{Interface utilisateur: }La présentation et les options initiales du chatbot, telles que les boutons de démarrage et les messages de bienvenue, sont configurées statiquement.
\end{itemize}

\subsection{Configuration dynamique}

\hspace{16pt}Concernant les aspects dynamiques de notre chatbot, il nous a été demandé de le rendre « entièrement » personnalisable, ce qui signifie:

\begin{itemize}
  \item \textbf{Construction dynamique: }Chaque message à l'écran doit être configurable via un tableau de bord, y compris les messages d'erreur.
  \item \textbf{Messages dynamiques: }Les messages doivent être modifiés en fonction du contexte.
  \item \textbf{Flux dynamique: }L'utilisateur doit avoir la possibilité de choisir entre où aller ensuite à partir de sa position actuelle (à l'exclusion bien sûr des flux dans lesquels nous demandons des informations critiques pour créer le compte de l'utilisateur)
  \item \textbf{Composant dynamique: }Les composants personnalisés doivent s'adapter aux entrées de l'utili-sateur et afficher les informations en conséquence.
  \item \textbf{Données dynamiques: }chaque information contenue par le chatbot provient de l'API qui connecte notre chatbot à la base de données.
\end{itemize}

\section{La base de donnéés}


\subsection{Création des entités}

\hspace{16pt}La création des entités dans la base de données est une étape cruciale pour structurer les informations de manière efficace et logique. Voici les principales entités créées pour ce projet:

\begin{itemize}
  \item \textbf{Utilisateur: }Cette entité représente les utilisateurs du système, avec des rôles distincts (patient ou administrateur). Elle stocke des informations telles que le nom, l'adresse e-mail, le rôle et les identifiants de connexion.
  \item \textbf{Medecin: }Cette entité contient les détails des médecins, y compris leur nom, leur spécialité, et leurs horaires de disponibilité. Chaque médecin est associé à une ou plusieurs spécialités.
  \item \textbf{Specialite: }Cette entité représente les différentes spécialités médicales disponibles dans la maison médicale. Elle est liée à l'entité Médecin pour indiquer les compétences de chaque médecin.
  \item \textbf{RendezVous: }Cette entité enregistre les rendez-vous programmés, y compris la date, l'heure, le patient le médecin assigné, etc. Elle est cruciale pour la gestion et la planification des consultations.
  \item \textbf{QuestionsDeSpécialité: }Cette entité stocke des questions spécifiques liées à chaque spécialité médicale. Elle aide à fournir des réponses précises via le chatbot en fonction des spécialités.
  \item \textbf{Reponse: }Cette entité regroupe les réponses données par les patients aux questions spécifiques liées aux spécialités médicales.
  \item \textbf{JourDeSemaine: }Cette entité représente les jours de la semaine et est utilisée pour définir les horaires de disponibilité des médecins.
  \item \textbf{Horaire: }Cette entité définit les heures de début et de fin des consultations pour chaque jour de la semaine. Elle est liée à l'entité Médecin pour indiquer les horaires de travail de chaque médecin.
  \item \textbf{Session: }Cette entité représente les créneaux horaires spécifiques disponibles pour les rendez-vous. Elle est utilisée pour planifier les consultations et éviter les conflits de planning.
  \item \textbf{FAQ: }Cette entité stocke les questions fréquentes et leurs réponses associées, utilisées par le chatbot pour fournir des informations instantanées aux utilisateurs.
  \item \textbf{Block: }Cette entité représente les blocs de contenu dans le chatbot. Chaque bloc peut contenir des informations ou des options de réponse pour les utilisateurs.
  \item \textbf{BlockOption: }Cette entité stocke les différentes options de réponse disponibles pour chaque bloc du chatbot. Elle permet de définir des chemins de conversation en fonction des choix des utilisateurs.
  % \item \textbf{ }
\end{itemize}

En structurant ces entités de manière logique et efficace, nous avons pu créer une base de données robuste et évolutive, capable de gérer les différentes facettes du projet, y compris la gestion des utilisateurs, des médecins, des rendez-vous, et des interactions du chatbot.

\subsection{Configuration d'API}

\hspace{16pt}La configuration de l'API est une étape essentielle pour permettre une communication efficace entre le frontend et le backend de notre application. Pour ce projet, nous avons utilisé Symfony avec API Platform pour créer une API RESTful. API Platform offre une grande flexibilité et permet de configurer facilement les opérations CRUD (Create, Read, Update, Delete), la validation des champs, la gestion des relations, les filtres et les groupes de sérialisation. Voici un aperçu de ces fonctionnalités:

\begin{itemize}
  \item \textbf{Opérations CRUD Configurables: }API Platform facilite la configuration des opérations CRUD pour chaque entité. Par défaut, les opérations GET, POST, PUT, DELETE sont disponibles, mais elles peuvent être personnalisées ou restreintes selon les besoins.
  \item \textbf{Validation des Champs: }API Platform intègre le composant de validation de Symfony, ce qui permet de définir des règles de validation directement dans les entités.
  \item \textbf{La gestion des relations entre les entités est simplifiée avec API Platform. Les relations peuvent être configurées pour être exposées via l'API.}
  \item \textbf{API Platform permet d'ajouter facilement des filtres pour les opérations GET. Ces filtres permettent de rechercher et de trier les données de manière flexible.}
  \item \textbf{Groupes de Sérialisation: }Les groupes de sérialisation permettent de contrôler les données qui sont exposées via l'API. Cela est utile pour restreindre les champs visibles selon le contexte (lecture, écriture, etc.).
\end{itemize}

\section{Le Dashboard}

\subsection{La gestion des rendez-vous}

\hspace {16pt}La gestion des rendez-vous est une fonctionnalité essentielle du dashboard, permettant aux administrateurs et aux médecins de suivre et de gérer efficacement les consultations des patients. Voici les principales caractéristiques de cette fonctionnalité:

\begin{itemize}
  \item \textbf{Vue Calendrier: }Une interface calendrier permet de visualiser les rendez-vous par jour, semaine ou mois. Les rendez-vous sont colorés en fonction de leur statut (confirmé, en attente, annulé).
  \item \textbf{Confirmation et Modification: }Les administrateurs peuvent confirmer, modifier ou annuler des rendez-vous directement depuis le dashboard.
\end{itemize}

\subsection{La gestion d'équipe medicale}

\hspace{16pt} La gestion de l’équipe médicale est une autre fonctionnalité clé du dashboard, assurant une organisation optimale des ressources humaines et des services offerts. Voici comment cette fonctionnalité est mise en œuvre:

\begin{itemize}
  \item \textbf{Liste des Médecins: }Une liste détaillée des médecins avec leurs spécialités, disponibilités et coordonnées. Les administrateurs peuvent facilement ajouter, modifier ou désactiver des médecins.
  \item \textbf{Planification des Horaires: }Un module de planification permet de définir les horaires de travail des médecins.
  \item \textbf{Gestion des Spécialités: }Les administrateurs peuvent gérer les spécialités médicales, en ajoutant ou supprimant des spécialités et en assignant des médecins aux différentes spécialités.
  \item \textbf{Rapports et Statistiques: }Des rapports détaillés sur les activités des médecins, incluant le nombre de consultations, le taux de satisfaction des patients, et les performances par spécialité.
\end{itemize}

\break

\subsection{La gestion du Chatbot}

\hspace{16pt} La gestion du chatbot depuis le dashboard permet de maintenir et d’améliorer en continu l’interaction avec les patients. Voici les fonctionnalités principales de cette section:

\begin{itemize}
  \item \textbf{Configuration des Réponses: }Les administrateurs peuvent ajouter ou modifier les réponses prédéfinies du chatbot pour les questions fréquentes. Cela inclut la mise à jour des informations sur les horaires, les services et les procédures.
  \item \textbf{Personnalisation: }Des options de personnalisation permettent de modifier le comportement du chatbot, incluant les messages de bienvenue.
\end{itemize}

\newpage

\section*{Conclusion}

Lorem ipsum dolor sit amet, consectetur adipiscing elit. Praesent nec dapibus justo. Donec sagittis vulputate ante sed porttitor. Suspendisse sit amet nisl massa. Curabitur nec nisl condimentum, egestas ex vitae, dapibus enim. Etiam iaculis, erat faucibus pellentesque sagittis, nisi justo sollicitudin nibh, et condimentum augue massa non turpis. Proin commodo enim fermentum suscipit condimentum. Maecenas molestie, dui nec vestibulum rhoncus, arcu nisl faucibus neque, a ornare nisi massa ac eros. Aenean id velit sit amet lacus mattis varius. Donec fringilla massa sed nisi eleifend, a aliquet mi tempus. Nunc posuere euismod est, nec tristique augue lobortis non. Sed sodales sem ut metus tempus ullamcorper.

\pagebreak
