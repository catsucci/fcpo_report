\chapter*{Résumé}
\addcontentsline{toc}{chapter}{Résumé}

\hspace{16pt}Mon rapport de stage de fin d’études, sous la direction de Soufyane BOUKHRISS, porte sur le développement d’un chatbot et d’un système de gestion des rendez-vous pour une maison médicale. Réalisé au sein de l'agence digitale FCPO du 8 avril 2024 au 8 juin 2024, ce stage visait à obtenir le Diplôme Universitaire de Technologie (DUT) en Génie Informatique.

Le projet avait pour objectif de développer un chatbot pour une maison médicale afin d'automatiser les réponses aux questions fréquentes des patients et de faciliter la prise de rendez-vous. Ce projet répondait aux besoins croissants de digitalisation dans le secteur médical, visant à améliorer l'efficacité et la qualité du service aux patients.

Pendant ce stage, plusieurs concepts théoriques ont été approfondis, notamment l'architecture des applications web utilisant le modèle MVC (Model-View-Controller) avec Symfony et API Platform pour structurer les applications web, ainsi que la conception et la gestion de bases de données relationnelles avec MySQL.

Des compétences pratiques essentielles ont été développées, incluant le développement frontend avec la création d'interfaces utilisateur interactives en utilisant React, et le développement backend avec Symfony et API Platform pour créer des APIs RESTful. La gestion de version avec Git et GitHub a également été une compétence clé acquise, permettant une gestion efficace de la version et une collaboration en équipe. L’optimisation de l’environnement de développement avec des outils comme Arch Linux, Neovim, Alacritty, ZSH, tmux, et fzf a également été significative.

Le stage a offert une perspective précieuse sur le fonctionnement et la culture d’entreprise. La collaboration avec une équipe de développeurs, l’utilisation de méthodologies agiles pour la gestion de projet comme les sprints et les réunions quotidiennes, et l'adaptation aux exigences des clients ont été des aspects cruciaux. La compréhension de l’importance de répondre aux besoins des clients et de s’adapter rapidement aux changements de spécifications a été renforcée, tout en intégrant les valeurs et pratiques professionnelles de FCPO.

Le développement du chatbot a présenté plusieurs défis, tels que la maîtrise de nouvelles technologies comme Symfony, API Platform et React, la configuration complexe pour une réponse personnalisée aux interactions des utilisateurs, et la résolution de problèmes techniques liés à des drivers manquants par l’installation de packages et la vérification systématique des dépendances. Des problèmes de configuration et de gestion des domaines ont également été rencontrés, nécessitant une transition fluide entre les environnements de développement et de production.

En conclusion, le stage chez FCPO a été une expérience riche et formatrice, consolidant des connaissances théoriques et des compétences pratiques. Il a offert une immersion dans un environnement professionnel exigeant, développant des compétences en collaboration d’équipe et en gestion de projet, préparant ainsi efficacement à de futurs défis professionnels.

% \noindent\rule[2pt]{\textwidth}{0.5pt}
%
% {\textbf{Mots clés :}}
% xxx, xxx, xxx, xxx.
% \\
% \noindent\rule[2pt]{\textwidth}{0.5pt}
