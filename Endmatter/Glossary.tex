\chapter{Glossaire}
\label{chap:Glossary} 


\hspace{16pt}\textbf{Chatbot: }Un programme d'intelligence artificielle qui simule une conversation humaine.\\

\textbf{Arch Linux: }Une distribution Linux simple et légère.\\

\textbf{KISS: }Keep It Simple, Stupid (Garde-le simple, stupide), un principe de conception affirmant que les systèmes fonctionnent mieux s'ils sont simples.\\

\textbf{Alacritty: }Un émulateur de terminal rapide et multiplateforme.\\

\textbf{Zsh (Z Shell): }Une version étendue du Bourne Shell avec de nombreuses améliorations.\\

\textbf{Neovim (Neo-Vim): }Un éditeur de texte extensible et hautement configurable basé sur Vim.\\

\textbf{GitHub: }Une plateforme web utilisée pour le contrôle de version.\\

\textbf{Postman: }Une plateforme de collaboration pour le développement d'API.\\

\textbf{Figma: }Un outil de design web pour la conception d'interface et le prototypage.\\

\textbf{\LaTeX: }Un système de composition de documents souvent utilisé pour la documentation technique et scientifique.\\

\textbf{Eraser: }Pourrait faire référence à un outil utilisé dans un contexte spécifique, non défini explicitement.\\

\textbf{API Platform: }Une puissante plateforme pour créer des projets orientés API.\\

\textbf{REST: }Représentation État Transfert, un style architectural pour concevoir des applications en réseau.\\

\textbf{CRUD: }Créer, Lire, Mettre à jour, Supprimer, les quatre fonctions de base du stockage persistant.\\

\textbf{MVC: }Modèle-Vue-Contrôleur, un modèle architectural pour implémenter des interfaces utilisateur.\\

\textbf{React: }A JavaScript library for building user interfaces, maintained by Facebook.\\

\textbf{Symfony: }A PHP framework for web applications.\\

\pagebreak
